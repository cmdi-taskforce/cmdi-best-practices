\section{Outline}\label{outline}

The first two sections of the CMDI Best Practice guide follow the
lifecycle of Component Metadata, i.e.~modelling CMD and authoring CMD records. Best practices are given for various CMD constructs encountered at the different levels. Both these sections also provide guidelines regarding the workflow, e.g. (sequences of) actions to take and tools to use, and indicators of potential problems (also known as `smells').

\begin{workinprogress}
\bptodo{TODO: Finish this (Alex)}

\section{Outline}\label{outline}

This guide is split into two major parts. The first chapter follows the lifecycle of Component Metadata, i.e.~modelling CMD and authoring CMD records. Best practices are given for various CMD constructs encountered at the different levels. Both these sections also provide guidelines regarding the workflow, e.g. (sequences of) actions to take and tools to use, and indicators of potential problems (also known as `smells').

However, there are some approaches and problems, which can be considered ``crosscutting concerns'', having aspects that are not limited to either modelling or authoring. These are covered in the second chapter.

\end{workinprogress}
