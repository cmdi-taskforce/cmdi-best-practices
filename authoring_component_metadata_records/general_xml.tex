\subsection{General XML}\label{general-xml}
\editor{Menzo}

\label{x1}
X1: Include a reference to the profile XSD generated by the Component Registry

\priority{high}\clarinb{6.6}\checker{CMDI Instance Validator}

For each profile stored in the Component Registry a dynamically generated XSD is available. The URL of this XSD is available in the Info dialogue of a profile and should be included in the schemaLocation attribute on the CMD root element. This enables validation of a CMD record by general XSD validators, including the CMDI Validator. The Component Registry URL should be used as it ensures that fixes in the
transformation from a profile specification into an XSD are included in the validation process.

\label{x2}
X2: Use common namespace prefixes

\priority{low} \bptodo{TODO: check: CMDI Instance Validator

\url{https://github.com/clarin-eric/cmdi-toolkit/issues/9}}

Namespace prefixes are officially just syntactic sugar in XML, i.e., provide a convenient shortcut. However, using common prefixes enable users to quickly assess the scope of an element. The CMDI 1.2 specification recommends the following prefixes for the namespaces URIs in CMDI as described in Table~\vref{table:namespaces}.

\begin{sidewaystable}
\caption{Namespaces}
\label{table:namespaces}
\begin{tabu} to 1.0\textwidth {|X[1,l]|X[2,l]|X[1,l]|X[1,l]|}
    \hline
     \textbf{Prefix} & \textbf{Namespace Name} & \textbf{Comment} & \textbf{Recommended Syntax} \\ \hline
     cmd & \makecell[l]{\tt{http://www.clarin.eu/cmd/1}} & CMDI instance (general/envelope) & prefixed \\ \hline
     cmdp & \makecell[l]{\tt{http://www.clarin.eu/cmd/1} \\\tt{/profiles/\{profileid\}}} & CMDI payload (profile specific) & prefixed \\ \hline
     cue & \makecell[l]{\tt{http://www.clarin.eu/cmd/cues/1}} & Cues for tools & prefixed \\ \hline
     xs & \makecell[l]{\tt{http://www.w3.org/2001/XMLSchema}} & XML Schema & prefixed \\ \hline
     xsi & \makecell[l]{\tt{http://www.w3.org} \\\tt{/2001/XMLSchema-Instance}} & XML Schema Instance & prefixed \\ \hline
\end{tabu}
\end{sidewaystable}
See section \nameref{authoring-workflow} regarding validation, which implies well-formed XML.

X3: Use UTF-8 encoding

\priority{high}

\bptodo{Menzo: split: specify encoding used (high), use UTF-8 (high)}

The encoding of a CMD record, i.e., XML documents in general, doesn't have to be stated explicitly. It can be provided in various, possibly conflicting ways: via a Byte Order Marker (BOM), in the XML declaration of the document or a HTTP header. The best practice is to align all these methods to express an UTF-8 encoding, but include at least the XML declaration to indicate the encoding used.
